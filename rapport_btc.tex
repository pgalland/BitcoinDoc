\documentclass[11pt,a4paper]{article}
\usepackage[top=80px, left=70px, right=70px]{geometry}

\usepackage[T1]{fontenc}
\usepackage[utf8]{inputenc}
\usepackage{lmodern}
\usepackage[francais]{babel}
\usepackage{graphicx}
\usepackage{float}

\usepackage{algorithm}
\usepackage{algorithmic}

\usepackage[table]{xcolor}

\renewcommand{\algorithmicrequire}{\textbf{Input:}}
\renewcommand{\algorithmicensure}{\textbf{Output:}}

\title{Comprendre Bitcoin\\
Rapport de projet INFRES357
}

\author{Pierre Galland, Benoît de Laitre}

\begin{document}
\maketitle

\tableofcontents

\newpage

\section{Briques de base}

Le protocole de Bitcoin utilise plusieurs briques de bases de la cryptographie : le chiffrement asymétrique, la signature et le hachage. C'est d'ailleurs là que réside l'intérêt de ce protocole, c'est un assemblage astucieux de briques qui existent depuis de nombreuses années et qui sont très bien connues, pourtant cet assemblage crée une technologie totalement nouvelle.

\subsection{Chiffrement asymétrique}
Le principe du chiffrement asymétrique est que les clefs vont par paire, une clef publique et une clef privé $(K_{Pub}, K_{Pri})$. Il est possible de chiffrer un message avec la clef privé et alors le message chiffré ne pourra être déchiffré qu'avec la clef publique correspondante. De même on peut chiffrer un message avec la clef publique et alors il ne pourra être déchiffré qu'avec la clef privé correspondante. Considérons l'exemple classique de Bob et Alice, ils possède chacun une paire clef publique/clef privée. La paire de clefs de Bob est 
$(K_{Pub}^{B}, K_{Pri}^{B})$ et celle d'Alice est $(K_{Pub}^{A}, K_{Pri}^{A})$.\\\\

Si Bob veut envoyer un message $mess$ à Alice qu'elle seule pourra lire, alors il chiffre son message avec le clef publique d'Alice, et il obtient le message chiffré $*mess*$ :

$$chiffrer(mess, ~K_{Pub}^{A}) \rightarrow *mess*$$

Il envoie alors ce message $*mess*$ à Alice, qui quand elle le reçoit le déchiffre avec sa clef privée $K_{Pri}^{A}$ :

$$déchiffrer(*mess*, ~K_{Pri}^{A}) \rightarrow mess$$

Si Alice essayait de déchiffrer le message $*mess*$ avec une autre clef que sa clef privée $K_{Pri}^{A}$ alors cela ne marcherait pas, elle obtiendrait n'importe quoi (une suite de symbole qui n'a rien à voir avec le message $mess$). Donc comme on considère qu'Alice est la seule à connaître sa clef privée, elle est la seule à pouvoir déchiffrer le message.

$$déchiffrer(*mess*, ~K_{Pri}^{autre}) \rightarrow n'importe~quoi$$


\subsection{Signature}
\subsection{Hachage}

\section{Un paiement Bitcoin}

\section{Diffuser les paiements}

\section{Empêcher le double-spend}

\section{Le minage}
\section{Les différents types de clients}
\subsection{Full clients}
Full block chain,... Essential to running the network.
\subsection{Headers-only clients}
Download and store only block headers. Must trust people running full clients.
Procedure "simple payment verification", also described in the original Bitcoin paper.
\subsection{Signing-only clients}
Ne récupère que ce qui le concerne personnellement.
Il faut un certain niveau de confiance dans le serveur (qui peut nous envoyer de fausses infos, et par exemple nous faire croire qu'on a reçu de l'argent alors que non), mais le serveur ne peut pas nous voler de l'argent (nous faire signer des transactions contre notre volonté) (le client peut être une appli PC, mobile ou web ; pour une appli Web, la signature est faite dans le browser, donc les clefs ne sont pas envoyées).

\subsection{Thin clients}
Mt.Gox,...
Le client ne détient pas ses clefs et ne signe pas lui-même ses transactions, il demande à un serveur de faire cela pour lui. En ce sens, c'est exactement la même chose qu'une banque.
\bibliography{biblio}{}
\bibliographystyle{plain}

\end{document}