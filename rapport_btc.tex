\documentclass[11pt,a4paper]{article}
\usepackage[top=80px, left=70px, right=70px]{geometry}

\usepackage[T1]{fontenc}
\usepackage[utf8]{inputenc}
\usepackage{lmodern}
\usepackage[francais]{babel}
\usepackage{graphicx}
\usepackage{float}

\usepackage{algorithm}
\usepackage{algorithmic}

\usepackage[table]{xcolor}

\renewcommand{\algorithmicrequire}{\textbf{Input:}}
\renewcommand{\algorithmicensure}{\textbf{Output:}}

\title{Comprendre Bitcoin\\
Rapport de projet INFRES357
}

\author{Pierre Galland, Benoît de Laitre}

\begin{document}
\maketitle

\tableofcontents

\newpage

\section{Briques de base}

Le protocole de Bitcoin utilise plusieurs briques de bases de la cryptographie : le chiffrement asymétrique, la signature et le hachage. C'est d'ailleurs là que réside l'intérêt de ce protocole, c'est un assemblage astucieux de briques qui existent depuis de nombreuses années et qui sont très bien connues, pourtant cet assemblage crée une technologie totalement nouvelle.

\subsection{Chiffrement asymétrique}
\subsection{Signature}
\subsection{Hachage}

\bibliography{biblio}{}
\bibliographystyle{plain}

\end{document}